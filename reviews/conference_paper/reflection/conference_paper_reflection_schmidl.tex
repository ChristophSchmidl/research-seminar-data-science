\documentclass[a4paper]{article}

\usepackage{enumitem}
\usepackage{natbib}
\usepackage[english]{babel}
\usepackage{amsmath}
\usepackage{amssymb}
\usepackage{dsfont}
\usepackage{graphicx}
\usepackage{listings}
\usepackage[hyphens]{url}
\usepackage{titling}
\usepackage{varwidth}
\usepackage{hyperref}
\usepackage{color} %red, green, blue, yellow, cyan, magenta, black, white
\definecolor{mygreen}{RGB}{28,172,0} % color values Red, Green, Blue
\definecolor{mylilas}{RGB}{170,55,241}


\usepackage{geometry}
 \geometry{
 a4paper,
 total={165mm,257mm},
 left=20mm,
 top=20mm,
 }

\title{Research Seminar in Data Science\\Review of a conference paper\\Reflection}
\author{
  Christoph Schmidl\\ s4226887\\      \texttt{c.schmidl@student.ru.nl}
}
\date{\today}
\date{\today}

\begin{document}
\maketitle





\section{Reflection: Causal Inference in Time Series via Supervised Learning}

My review scores range between 6 and 7 for the individual grading criteria, where I chose to give a 6 for the overall score and a 4 for my level of confidence on my assessment. Therefore I would accept the paper. The senior PC members gave an overall score of 8 and the three remaining PC members gave a score of 7, 5 and 8 respectively. This makes three votes for acceptance for this paper and one vote for not accepting, given that my assessment does not count. I am glad that my general intuition, after acquiring a brief understanding of the presented paper topic goes into the same direction as the PC members. However, based on my level of confidence on this topic and my expertise, I misjudged the relevance and significance of this paper and decided to make a more conservative score compared to the PC members.\\
When I looked into the disussion between the reviewers, I discovered a broad variety of discussion points which are different from mine. While I was more concerned and busy with understanding the general idea and approach of the paper, I overlooked some basic grading criteria which I could have spotted even without understanding the technical details. The discussion was mostly about the application to real-world data and that sufficient real-world examples are missing in the paper. Another big point is the lack of missing performance evaluation which I also could have spotted but totally overlooked. I often made comments with respect to a cited paper which was chosen as the main model to compare to. The cited paper seems to be the state-of-the-art approach and was beaten by the proposed paper. Therefore I concluded that the paper is significant which is not wrong in general. The PC members not only discussed about the cited paper and the used model but also that there's a small lack of an adequate comparison between similar models which makes the comparison slightly biased. I totally agree with that and that is another point that I could have spotted. My review seems to be more general and positive, while the reviews of the PC members are very specific and critical. Although I missed many points compared to the PC members, I recognized the proposed method as a novel approach and also thought that the feature engineering steps are sophisticated and elegant.
I fully agree with the final decision of the PC members to accept the paper and the report of the meta-reviewer.\\
I discovered that the general approach to review a paper is quite different between the different PC members. Some members prefer to give a small summary at the beginning of their review and then go into the details. Some reviewers only address the points that they disagree with, while other reviewers write both positive and negative comments. I think looking into the actual disucssion was insightful for me and changed my perspective on how to read and review papers.



\end{document}
