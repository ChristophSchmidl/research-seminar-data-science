\documentclass[a4paper]{article}

\usepackage{enumitem}
\usepackage{natbib}
\usepackage[english]{babel}
\usepackage{amsmath}
\usepackage{amssymb}
\usepackage{dsfont}
\usepackage{graphicx}
\usepackage{listings}
\usepackage[hyphens]{url}
\usepackage{titling}
\usepackage{varwidth}
\usepackage{hyperref}
\usepackage{color} %red, green, blue, yellow, cyan, magenta, black, white
\definecolor{mygreen}{RGB}{28,172,0} % color values Red, Green, Blue
\definecolor{mylilas}{RGB}{170,55,241}


\usepackage{geometry}
 \geometry{
 a4paper,
 total={165mm,257mm},
 left=20mm,
 top=20mm,
 }

\title{Research Seminar in Data Science\\Review of a conference paper}
\author{
  Christoph Schmidl\\ s4226887\\      \texttt{c.schmidl@student.ru.nl}
}
\date{\today}
\date{\today}

\begin{document}
\maketitle





\section{Causal Inference in Time Series via Supervised Learning}

\setlength{\tabcolsep}{0.5em} % for the horizontal padding
{\renewcommand{\arraystretch}{1.2}% for the vertical padding
\begin{table}[h]
\centering
    \begin{tabular}{|l|l|}
    \hline
    \textbf{Criteria} & \textbf{Score} \\ \hline
    Relevance & ~~ 6\\ \hline
    Significance & ~~ 7\\ \hline
    Originality & ~~ 6\\ \hline
    Technical Quality & ~~ 7\\ \hline
    Clarity and quality of writing & ~~ 7\\ \hline
    Scholarship & ~~ 7\\ \hline
     & \\ \hline
    Overall Score & ~~ 6 \\ \hline
    Confidence on your assessment & ~~ 4 \\ \hline
    \end{tabular}
\end{table}





\subsection{Main Review}

\subsection{Comments to Authors. (free text)}

\textbf{Relevance}:\\

The paper fully fits into the scope of the conference. Although the paper might only spark the interest of a minor part of the researchers based on the very specific topic that it addresses, the presented results in itself are interesting. Based on its topic specificity and the number of conference submissions, this paper may not get accepted. Previous relevant work has been cited properly and the amount of citations is solid. One suggestion would be to also look into the paper "Supervised Estimation of Granger-Based Causality between Time Series" by Benozzo et al. (2017) because it addresses a similar topic, although this is purely optional.\\

\noindent \textbf{Significance}:\\

The state-of-the-art approach "Towards a learning theory of cause-effect inference" by Lopez-Paz et al. (2015) has been chosen as the  baseline for later comparison. This makes sense because this approach is also using a supervised learning method for i.i.d. data called "Randomized Causation Coefficient (RCC)" which is using kernel mean embeddings to obtain features. The proposed method is beating the state-of-the-art results which makes this paper likely to get read and cited. Given that this paper represents the "new" state-of-the-art approach, it would be likely to have a lasting impact. The paper adresses a very specific problem and is therefore not important for most researchers at the conference. The paper opens new research directions, e.g., in terms of performance differences of other classification models besides random forest operating on the proposed feature vectors.\\

\newpage

\noindent \textbf{Originality}:\\

The combination of different feature engineering steps (using kernel mean embeddings to map distributions into a reproducing kernel hilbert space, applying maximum mean discrepancy as a distance measure in that space and later on using random fourier features) makes this paper distinct and novel from the work of Lopez-Paz et al. (2015), although I do not know if this approach is novel in general.\\

\noindent \textbf{Technical Quality}:\\

I could not find any obvious flaws in the conceptual approach. The claims are well-supported by experimental results and are convincing. The evaluation is based on comparing the proposed method with the state-of-the-art approach "RCC" by Lopez-Paz et al. (2015) and granger causality methods, namely the Vector Autoregressive Model, Generative Additive model and the kernel regression. The macro and micro-averaged F1-score is used as the evaluation metric. The evaluation is appropriate. The authors also show that their approach works with n-variate time series where n $>$ 3 although it seems to be rather restricted to trivariate time series. This is not a technical error, I just hoped the proposed method would work equally well with n-variate time series where n $>$ 3.\\


\noindent \textbf{Clarity and quality of writing}:\\

The paper is well organized and cleary written. The use of examples and figures is sufficient. I could not find any mistakes with regards to style, grammar, formatting or typos.\\

\noindent \textbf{Scholarship}:\\

The paper situates the state-of-the-art approach of Lopez-Paz et al.(2015) in a good way. Relevant papers were cited and well-known models were chosen for further comparison besides the Randomized Causation Coefficient (RCC). The experiments were chosen in a direct comparison towards RCC but also to traditional granger causality models like VAR, GAM, KER and to transfer entropy (TE).



\subsection{Confidential Comments (Not visible to the authors, only to other reviewers/program char)}








\end{document}
