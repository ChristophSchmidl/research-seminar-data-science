\documentclass[a4paper]{article}

\usepackage{natbib}
\usepackage[english]{babel}
\usepackage{amsmath}
\usepackage{amssymb}
\usepackage{dsfont}
\usepackage{graphicx}
\usepackage{listings}
\usepackage[hyphens]{url}
\usepackage{titling}
\usepackage{varwidth}
\usepackage{hyperref}
\usepackage{color} %red, green, blue, yellow, cyan, magenta, black, white
\definecolor{mygreen}{RGB}{28,172,0} % color values Red, Green, Blue
\definecolor{mylilas}{RGB}{170,55,241}


\usepackage{geometry}
 \geometry{
 a4paper,
 total={165mm,257mm},
 left=20mm,
 top=20mm,
 }

\title{Research Seminar in Data Science\\Review of a scientific presentation}
\author{
  Christoph Schmidl\\ s4226887\\      \texttt{c.schmidl@student.ru.nl}
}
\date{\today}
\date{\today}

\begin{document}
\maketitle

\textbf{Reviewed presentation:}

\begin{itemize}
	\item \textbf{Equivariant Neural Networks} by Taco Cohen on the first Deep Learning Nijmegen Metup on the 14th June 2018
\end{itemize}


\section{Equivariant Neural Networks}

\subsection{Summary}

Taco Cohen: https://tacocohen.wordpress.com/

Summary (as briefly as you can – two or three sentences)

\subsection{Evidence}

Harmonic Networks: Deep translation and rotation equivariance \cite{worrall2017harmonic}


Group equivariant convolutional networks \cite{cohen2016group}


Improved Semantic Segmentation for Histopathology using Rotation Equivariant Convolutional Networks \cite{winkens2018improved}


3D G-CNNs for Pulmonary Nodule Detection \cite{winkels20183d}


Evidence (what evidence is offered to support the claims made?)

\subsection{Strengths}

Strengths (what positive basis is there for listening to it?)

\subsection{Weaknesses}

Weaknesses

\subsection{Evaluation}

Evaluation (if you were running the workshop/conference/seminar where the presentation was held, would you recommend inviting the speaker again?)

\subsection{Comments on the quality of the presentation}

Comments on the quality of the presentation

\bibliographystyle{apa}
\bibliography{literature} 


\end{document}
